\providecommand{\main}{..}
\documentclass[\main/main.tex]{subfiles}

\begin{document}


\subsection{Matrices}
Matrices are \lq\lq a rectangular array of numbers whose dimension is $m \times n$'' \citep{Bradley1977}. If $m = n$, the matrix is said to be \textit{square}. 
Matrices will be represented by capital letters in bold (e.g. $\mathbf{P}$) with their entries in italic (e.g. $p_{ij}$ is the $(i,j)$ element of $\mathbf{P}$).\\
If the matrix has a row-to-column orientation, $i$ indexes the row and $j$ the column (e.g. $p_{12}$ represents the entry in the first row, second column). Conversely, when the matrix has a column-to-row orientation, $i$ indexes the column and $j$ the row (e.g. $p_{12}$ represents the entry in the first column, second row).

\noindent For instance:\\
\begin{center}
\begin{minipage}{.4\linewidth}
\begin{equation*}
\mathbf{P} =
\begin{pmatrix}
p_{11} & p_{12} & \cdots & p_{1n}\\
p_{21} & p_{22} & \cdots & p_{2n}\\
\vdots & & \vdots\\
p_{m1} & p_{m2} & \cdots & p_{mn}\\
\end{pmatrix}
\end{equation*}
\end{minipage}%
\begin{minipage}{.4\linewidth}
\begin{equation*}
\mathbf{P} =
\begin{pmatrix}
p_{11} & p_{21} & \cdots & p_{n1}\\
p_{12} & p_{22} & \cdots & p_{n2}\\
\vdots & & \vdots\\
p_{1m} & p_{2m} & \cdots & p_{nm}\\
\end{pmatrix}
\end{equation*}
\end{minipage}
   
\end{center}


\subsection{Vectors}
Vectors are special cases of matrices with dimension $k \times 1$ . Vectors will be denoted by lowercase bold letters, e.g. $\mathbf{v}$.
A k-dimensional vector $\mathbf{v}$ is defined as \lq\lq an ordered collection of k real numbers $v_1, v_2, . . . , v_k$'' \citep{Bradley1977}, and is written as:

\begin{equation}
\mathbf{v} =
\begin{pmatrix}
v_1\\
v_2\\
\vdots\\
 v_k
\end{pmatrix}
\end{equation}
Vectors are by default column vectors. To indicate horizontal vectors, we will use the transpose of $\mathbf{v}$, i.e.  $\mathbf{v}^T = ( v_1,v_2, \cdots, v_k)$.

\subsection{Matrix multiplication}


The product of an $m \times p$ matrix $\mathbf{A}$ and a $p\times n$ matrix $\mathbf{B}$ is defined to be \lq\lq a new $m \times p$  matrix $\mathbf{C}$, written $\mathbf{C}$ = $\mathbf{AB}$, whose elements $c_{ij}$ are given by'' \citep{Bradley1977AppliedProgramming}:
\begin{equation}
    c_{ij} = \sum_{k=1}^p a_{ik}b_{kj}
\end{equation}\\

\noindent For instance:
\begin{equation*}
\begin{pmatrix}
a_{11} & a_{12} & \cdots & a_{1p}\\
a_{21} & a_{22} & \cdots & a_{2p}\\
\vdots & & \vdots\\
a_{m1} & a_{m2} & \cdots & a_{mp}\\
\end{pmatrix}
\begin{pmatrix}
b_{11} & b_{12} & \cdots & b_{1n}\\
b_{21} & b_{22} & \cdots & b_{2n}\\
\vdots & & \vdots\\
b_{p1} & b_{m2} & \cdots & b_{pn}\\
\end{pmatrix}
= 
\end{equation*}
\begin{equation*}
\begin{pmatrix}
c_{11} = a_{11}b_{11} + a_{12}b_{21} + ...+ a_{1p}b_{p1} &  & \cdots & \\
 &  & \cdots & \\
\vdots & & \vdots\\
 &  & \cdots & \\
\end{pmatrix}
\end{equation*}

\noindent Note that the product of two matrices is defined only if the number of columns of $\mathbf{A}$ is equal to the number of rows of $\mathbf{B}$. Morevoer, matrix multiplication is in general not commutative, i.e. $\mathbf{AB} \neq \mathbf{BA}$.




\end{document}