\providecommand{\main}{..}
\documentclass[\main/main.tex]{subfiles}

\begin{document}


\section{Stochastic processes}

\subsection{Theory of stochastic processes}

Stochastic processes have been developed as a tool to understand randomness in process, where by random process we mean a \lq\lq sequence of events in which each step follows from the last after a random choice" \citep{Holmes2015}. Stochastic process is essentially a \lq\lq fancy name'' for any family of random variables \citep{Chung2010}.\\
\cite{Ching2006} illustrate the idea of stochastic processes through a simple and an intuitive example:\\

\begin{small}
Suppose you live in a town where there exist two supermarkets: Wellcome and Park'n.
By reading the results of a marketing research, you know that a current customer of Wellcome supermarket will go to Park’n for her next shopping
with probability $\alpha \geq 0 $. Conversely, a current customer of Park'n supermarket may switch to Wellcome in her next shopping with a probability of $\beta \geq 0$. Let $X_t$ be a random variable representing a consumer's choice. At each time, $X_t$ can either take value Wellcome (W) or Park'n (P). $\{X_t, t \in T.\}$ is said to be a stochastic process whose outcome is unknown until we observe one of the many possible realisations, e.g. $X_1 = W, X_2 = P, X_3 = W, X_4 = W, ... $\\
\end{small}


\noindent To sum up, the idea is to investigate physical situations whose outcome cannot be predicted until it is observed, a so-called \lq\lq random experiment''. To help such investigation, we can define a function, called \lq\lq random variable'', that assigns a real number to each possible outcome of the random experiment. More formally, a random variable is a function from the sample space $\Omega $ (i.e. the set of possible outcomes of a random experiment) to the real numbers: $X : \Omega \rightarrow \mathds{R}$.

For instance, in the example of supermarkets mentioned above, we could define a random variable $X_t$ that takes value $0$ if $X_t = W$, and value $1$ if $X_t = P$.\\

\noindent As we have seen, the outcome of a single experiment can be studied via the notion of random variables.  Similarly, the outcome of a sequence of random experiments $X_1 = 0, X_2 = 1, X_3 = 0, X_4 = 0, ... $ can be studied via the notion of stochastic processes $  \{ X_t, t \in T\}$.  In simple parlance, a stochastic (or random) process is understood as \lq\lq an indexed sequence of random variables" \citep{Ching2006}.

The theory of stochastic processes concerns the \lq\lq structure of families of
random variables $X_t$ , where $t$ is a parameter running over a suitable index set $T$'' \citep{Karlin1975}.
Usually $t$ usually denotes time, meaning that, at every time $t$ in the set $T$, a random number $X_t$ is observed.
Some authors use an equivalent notation for a stochastic process $\{X_n, n \in N\}$, where the index $t$ is replaced by $n$.\\


See \cite{Karlin1975} for a detailed discussion of stochastic process. Just to mention some essential terminology, we will assume that every random variable $X_t$ takes a real value in the set $S$, with $S \subseteq \mathds{R}$. The set $S$ is referred as the \textit{state space}. If this set is finite or countable, the state space $S$ is said to be \textit{discrete}, otherwise it is said to be \textit{continuous}. A further distinction is made between \textit{discrete-time} stochastic process and \textit{continuous-time} ones. In the former, the set $T$ is finite or countable, i.e. $T = \{ 0,1,2,..\}$ while in the latter $T$ is not finite or countable, i.e $T = [0,\infty)$, or $T = [0,K]$ for some K. In this case $X_t$ is allowed to change at every instant in time \citep{Holmes2015}.


\subfile{sections/3a.markov_chain}
\subfile{sections/3b.markov_chain_with_rewards}

\end{document}