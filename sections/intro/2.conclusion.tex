%\providecommand{\main}{..}
%\documentclass[\main/main.tex]{subfiles}


\begin{document}
\chapter{Conclusion}



In this analysis we constructed a matrix population model following the theoretical framework set up by \cite{Caswell2018}. Our goal was to estimate healthy and unhealthy longevity for male and female population in some selected EU countries. The population is assumed to be closed, meaning that we do not consider any possible effect of immigration.\\

The focus of the study was on possible differentials in mortality and disability driven by differences in income. The computation was done for a sample of individuals aged above 50 years old (first specification) or above 65 years old (second specification), using data obtained from EASYSHARE. Results were similar in both specifications. However, we eventually decided to presents results for individuals aged above 65 since their income is likely to be less affected by \lq\lq income trajectories'' (which could have a strong effect on healthy and unhealthy longevity) and therefore should be more robust.\\

One important remark is that the projected life expectancy of individuals is based on parameters estimated from the observed data and that the transition probabilities employed in the model are assumed not to vary in the future, i.e. it is assumed that disability rates or income inequality are not subject to any trend.
Uncertainty in the outcomes hence arises also from the assumptions on these future trends \citep{ONeill2001AProjections}.\\


For example, when we project the life course of an individual aged 65 today, we are assuming that her probability of transitioning to any state in the future, e.g. to survive from age 80 to age 81, is the same as the the probability of a current 80-year old survey participant. This is an important assumption that is unlikely to represent the actual experience of any individual. 
For this reason, one should remember that the results do not represent the true life history of a cohort over time but apply to a synthetic cohort. Nevertheless, such projections may be helpful in describing historical changes in population size and for understanding current inequalities.\\


The second important methodological observation is that the level of analysis was mainly statistical and descriptive. This is sometimes referred as the first step or the \lq\lq discovery'' phase of the study of population change, i.e. a stage of \lq\lq demographic inquiry aimed at producing solid evidence on population trends and patterns, as well as their associations across time and space'' \citep{Billari2015IntegratingChange}. We are leaving to future investigations the goal to explore the driving factors of this phenomenon, identifying the causal patterns and the interactions among individuals that are generating the observed association.\\






In summary, results show an effect of an income gradient on both total life expectancy and on time spent 
in disability. Interestingly, there is some relevant variability across the EU countries considered in the study.
This suggests that, in some settings, higher income groups benefit, on average, from an increase in life expectancy and that their ageing is less affected by disability. Moreover, when looking at synthetic measures on projected \lq\lq life time'' income, such as the Theil or the Gini index, the estimated inequality in life expectancy translates into higher values of the coefficients, i.e. into a higher level of inequality within the country. As already mentioned, these are projected values which, however, could be useful in order to reason about the current situation. In fact, countries that are typically regarded as being very \lq\lq equal'', such as Denmark, might actually conceal a deeper level of inequality.\\



One limitation of this research is that, due to the type of information contained in SHARE, we have been unable to investigate two other important issues related to the topic of healthy longevity: firstly, whether the gap between life expectancy and income levels is widening over time (as it has been shown for the United States by \cite{Auerbach2017}), and secondly, whether there is a positive or negative relationship between life expectancy and time spent in disability.\\

Indeed, an increasing trend in life expectancy could have two opposite effects on the total time spent in disability: on the one hand, a decrease in mortality rates could lead to a lengthening of the life of people with disabilities, while on the other hand, the same factors associated with a decrease in the mortality rate (such as progress in health, nutrition, environmental factors, etc.) could also reduce disability rates of the ageing population.
Whether one effect or the other prevails could be of great concern for the planning of the health care budgets. 




\end{document}