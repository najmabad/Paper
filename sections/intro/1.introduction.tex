%\providecommand{\main}{..}
%\documentclass[\main/main.tex]{subfiles}


\begin{document}


\chapter{Introduction}

Over the last two centuries, world life expectancy has more than doubled, and, by looking at its trajectories, it seems that is has not reached its maximum yet \citep{Oeppen2002}.
Similarly, mortality rates are decreasing in all G7 countries \footnote{Canada, France, Germany, Italy, Japan, UK, US} \citep{Tuljapurkar2000}.
European countries are not an exception: at the beginning of the 20th century, life expectancy at birth was approximately 50 years in the vast majority of European states. After one century, it has increased up to 81 years for females and 75 years for men \citep{beer2002}.\\

Higher life expectancy and low birth rates are the markers of a demographic transition towards a much older population, which will arguably translate into a series of economic and political challenges \citep{Lee2003}.
For instance, disability and health-related benefits, such as illness insurance and long-term care, become increasingly important in periods of demographic ageing and even more so in countries with a better educated public which spurs an increase in the demand for health insurance \citep{haberman1999}.\\

Several studies on the social determinants of life expectancy - see, for instance, the articles of \cite{Chetty2016}, \cite{Mackenbach2013}, \cite{Duggan2008a}, among many others -  have pointed out at a positive relationship between income and life expectancy. \\
\indent The debate on this subject has developed along various dimensions. 
One major strand concerns the relationship between life expectancy and national per capita income, a hypothesis introduced by \cite{Preston1975a} in his seminal paper and since then extensively investigated. For example, there is a series of nice commentaries published on the International Journal of Epidemiology written by \cite{Bloom2007}, \cite{Leon2007}, \cite{Wilkinson2007}, \cite{Mackenbach2013a}, among the many publications on the topic.\\


\indent A second strand of the debate has developed starting from the work of \cite{Wilkinson1992}, which gave rise to the so-called \lq\lq inequality hypothesis'' and has shifted the focus from the average national income to the distribution of income, i.e. to the level of inequality within a given country.  
This idea has also being thoroughly studied by scholars, as highlighted in \cite{Kim2017b}'s review.\\


Overall, most research has found that higher income can help to improve life expectancy, although there is not yet a consensus on the main determinants of the observed correlation \citep{Jetter2016}.
One interesting point is that not only a gap in life expectancy by SES appears to exist but also it seems to be currently widening, as research by \cite{Auerbach2017} pointed out. This result has various implications for government entitlement programmes, such as pension systems and health care services.\\



In this study, we focus both on average life expectancy by income and on differences in healthy and unhealthy longevity. In other words, we attempt to estimate whether, and to what extent, the additional years of life, added by the declining trend in mortality of the past centuries, are free of disability and or not. Moreover, we are particularly interested in how differentials in healthy life expectancy by income are spread within and across countries. \\

This, in turn, has an important implication for the redistributive scope of welfare and pension programs: assuming that there is an income gradient that harms the poorest groups, generating a shorter life span and more years of disability, we would find ourselves in a setting in which the most disadvantaged receive pension benefits for fewer years and enjoy them to a lesser extent because of their physical illness.\\
Finally, years of disability directly translate into an increased burden for the working population, and therefore constitute an an important indicator for good planning of social expenditures. As a matter of fact, healthy life expectancy (HLE) is currently a widely used metrics of population health \citep{Riffe2017}.

\end{document}